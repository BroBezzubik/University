\addcontentsline{toc}{section}{Список использованных источников}
%\begingroup
\makeatletter \renewcommand\@biblabel[1]{#1.} \makeatother
\renewcommand\refname{Список использованных источников}
\begin{thebibliography}{00}
	\bibliographystyle{ugost2008}
	\bibitem{1}
	Sanborn. What We Do. 3D Visualization. 3D Cities. $//$ URL: $http://www.sanborn.com/3d-cities/$ (Дата обращения: 11.05.19)
	\bibitem{2}
	OpenStreetMap. $//$ URL: $https://www.openstreetmap.org/search?\\query=\%D0\%98\%D0\%B2\%D0\%B0\%D0\%BD\%D1\%82\%D0\\\%B5\%D0\%B5\%D0\%B2\%D0\%BA\%D0\%B0\#map=14/55.9717/37.9305$ (Дата обращения: 05.05.2019)
	\bibitem{3}
	ViziCities. $//$ URL: $https://github.com/UDST/vizicities$ (Дата обращения: 12.05.19)
	\bibitem{4}
	AUTODESK. $//$ URL: $https://www.autodesk.com$ (Дата обращения: 11.05.19)
	\bibitem{5}
	ArcGIS online. $//$ URL: $https://www.arcgis.com/index.html$ (Дата обращения: 12.05.19)
	\bibitem{6}
	Бурлуцкая А.Г., Локтионова Т.С. Обзор технических средств регулирования дорожного движения -- дорожных знаков. -- Статья в сборнике трудов конференции «Образование, наука, производство». -- С.942-944. -- 2015. $//$ URL: $https://elibrary.ru/item.asp?id=25571569$ (Дата обращения: 02.11.2019)
	\bibitem{7}
	Конвенция о Дорожных Знаках и Сигналах 1968 года. Европейское Соглашение, дополняющее Конвенцию, и Протокол о разметке дорог к Европейскому Соглашению (Сводный текст 2006 года). -- Организация Объединённых Наций. -- Нью-Йорк и Женева. -- 2007 год. -- 239 с. $//$ URL: $http://www.unece.org/fileadmin/DAM/trans/conventn/Conv\_road\_\\
	signs\_2006v\_RU.pdf$ (Дата обращения: 29.11.2019)
	\bibitem{8}
	Дорожные знаки к ПДД 2019 и их обозначения. $//$ URL: $https://ruspdd.ru/pdd/185-znaki/$ (Дата обращения: 27.11.2019)
	\bibitem{9}
	Как определить зону действия знака ПДД? $//$ URL: $https://avto-ur.com/na-doroge/kak-opredelit-zonu-dejstviya-znaka-pdd.html$ (Дата обращения: 19.11.2019)
	\bibitem{10}
	Кормилицына Л.В., Дуров Г.Р. Влияние дорожных знаков и разметки на безопасность дорожного движения. -- Дальний Восток. Автомобильные дороги и безопасность движения. -- С.188-191. -- 2017. $//$ URL: $https://elibrary.ru/item.asp?id=34948281$ (Дата обращения: 04.11.2019)
	\bibitem{11}
	Дорожная разметка РФ. $//$ URL: $https://pdd-russia.com/pdd-russia/znaki-i-razmetka/dorozhnaja-razmetka/russia.html$ (Дата обращения: 05.11.2019)
	\bibitem{12}
	Карманов Д.С., Марилов В.С. Современное состояние и перспективы развития платных дорог в России. -- Развитие теории и практики автомобильных перевозок, транспортной логистики. -- С.136-140. -- 2017. $//$ URL: $https://elibrary.ru/item.asp?id=32492205$ (Дата обращения: 24.10.2019)
	\bibitem{13}
	Алексеева Е.Ю. Обоснование размещения пунктов сбора платежей при проектировании платных автомобильных дорог. -- Дальний Восток. Автомобильные дороги и безопасность движения. -- С.184-187. -- 2016. $//$ URL: $https://elibrary.ru/item.asp?id=29023954$ (Дата обращения: 04.12.2019)
	\bibitem{14}
	Проезд по М-1. $//$ URL: $https://m-road.ru/road/$ (Дата обращения: 22.10.2019)
	\bibitem{15}
	Котов В.Е. Сети Петри. -- М. Наука. Главная редакция физико-математической литературы, 1984. -- 160 с.
	\bibitem{16}
	Виды дорожного покрытия автомобильных дорог. $//$ URL: $https://bouw.ru/article/vidi-dorozhnogo-pokritiya-avtomobilynih-dorog$ (Дата обращения: 09.12.2019)
	\bibitem{17}
	Физика. Механика. Равномерное прямолинейное движение. $//$ URL: $https://mnogoformul.ru/ravnomernoe-pryamolineynoe-dvizhenie$ (Дата обращения: 11.11.2019)
	\bibitem{18}
	Формулы ускорения в физике. $//$ URL: $https://spravochnick.ru/fizika/formuly\_uskoreniya\_v\_fizike/$ (Дата обращения: 11.11.2019)
	\bibitem{19}
	Формула обгона: как рассчитать расстояние, необходимое для манёвра? $//$ URL: $https://s30592398877.mirtesen.ru/blog/43920850036/Formula-obgona:-kak-rasschitat-rasstoyanie,-neobhodimoe-dlya-man$ (Дата обращения: 10.11.2019)
	\bibitem{20}
	Какой должна быть ширина дорожной полосы по ГОСТ. $//$ URL: $https://avtoedet.ru/shirina-dorozhnoy-polosyi-gost/$ (Дата обращения: 12.11.2019)
	\bibitem{21}
	Тормозной путь автомобиля от скорости и другие факторы (таблица). $//$ URL: $https://infotables.ru/avtomobili/1104-tormoznoj-put-avtomobilya-tablitsa\#hcq=JVJ0AHr$ (Дата обращения: 16.11.2019)
	\bibitem{22}
	Глубина протектора зимних и летних шин -- допустимая высота. $//$ URL: $ https://passus.ru/avto/glubina-protektora-zimnih-i-letnih-shin.html$ (Дата обращения: 18.12.2019)
	\bibitem{23}
	Что такое тормозной путь автомобиля и как его рассчитать? $//$ URL: $https://autochainik.ru/tormoznoy-put-avtomobilya.html$ (Дата обращения: 11.11.2019)
	\bibitem{24}
	Формула для расчета тормозного пути автомобиля. $//$ URL: $https://autopravilo.ru/pdd/formula-dlya-rascheta-tormoznogo-puti-avtomobilya.html$ (Дата обращения: 12.11.2019)
	\bibitem{25}
	Почему дорожное движение внезапно превращается в пробку. -- Перевод статьи (Оригинал: Benjamin Seibold. Why a Traffic Flow Suddenly Turns Into a Traffic Jam). $//$ URL: $https://habr.com/ru/post/449700/$ (Дата обращения: 17.10.2019)
	\bibitem{26}
	Дейт К.Дж. Введение в системы баз данных. $//$ 8-е издание. -- 2005. -- Часть 1. Основные понятия. -- С.46.
	\bibitem{27}
	Крёнке Д. Теория и практика построения баз данных. $//$ 8-е издание. -- 2003. -- Часть II. Моделирование данных. Глава 5. Реляционная модель и нормализация. -- С.166-201.
	\bibitem{28}
	Шустова И.Б. Данные, хранимые в виде графов. Области применения, перспективы, проблемы манипуляции. -- Статья в сборниках трудов конференции «Альманах научных работ молодых учёных университета ИТМО». -- С.274-277. -- 2017. $//$ URL: $https://elibrary.ru/item.asp?id=35403998$ (Дата обращения: 21.09.2019)
	\bibitem{29}
	Календарев А. NoSQL как он есть. -- Системный администратор. -- Номер 11 (132). -- С.51-55. -- 2013. $//$ URL: $https://elibrary.ru/item.asp?id=20466327$ (Дата обращения: 09.09.2019)
	\bibitem{30}
	Ткаченко А.В., Васильчикова А.В., Гришунов С.С. Обзор классов нереляционных баз данных. -- Электронный журнал: Наука, техника и образование. -- Номер 4 (9). -- С.81-85. -- 2016. $//$ URL: $https://elibrary.ru/item.asp?id=27664308$ (Дата обращения: 10.09.2019)
	\bibitem{31}
	Бочкарев П.В., Кононова М.В. Графовые модели данных. -- Теория. Практика. Инновации. -- Номер 12 (12). -- С.133-141. -- 2016. $//$ URL: $https://elibrary.ru/item.asp?id=27725671$ (Дата обращения: 13.09.2019)
	\bibitem{32}
	СУБД Postgres Pro. $//$ URL $https://postgrespro.ru/$ (Дата обращения: 27.12.2019)
	\bibitem{33}
	MySQL. $//$ URL: $https://www.mysql.com/$ (Дата обращения: 27.12.2019)
	\bibitem{34}
	What Is SQLite? $//$ URL: $https://www.sqlite.org/index.html$ (Дата обращения: 26.12.2019)
	\bibitem{35}
	Microsoft SQL Server. $//$ URL: $https://www.microsoft.com/ru-ru/sql-server$ (Дата обращения: 27.12.2019)
	\bibitem{36}
	Белоусов А.И., Ткачев С.Б. Дискретная математика. -- М.: МГТУ, 2006. -- Конечные автоматы и регулярные языки. -- С.460-579.
	\bibitem{37}
	Python. $//$ URL: $www.python.org$ (Дата обращения: 06.05.2019)
	\bibitem{38}
	The Py2neo v4 Handbook. $//$ URL: $https://py2neo.org/v4/$ (Дата обращения: 05.05.2019)
	\bibitem{39}
	Neo4j official page. Официальный сайт Neo4j. $//$ URL: $https://neo4j.com/$ (Дата обращения: 04.05.2019)
	\bibitem{40}
	Nikolaos Tsanakas, Joakim Ekström and Johan Olstam. Estimating Emissions from Static Traffic Models: Problems and Solutions. -- Journal of Advanced Transportation. -- Volume 2020. -- 17 pages. -- 2020. $//$ URL: $https://doi.org/10.1155/2020/5401792$ (Дата обращения: 13.03.2020)
	\bibitem{41}
	Fang Zong, Meng Zeng, Wei Zhong and Fengrui Lu. Hybrid Path Selection Modeling by Considering Habits and Traffic Conditions. -- IEEE Access. -- Volume 7. -- Pages 43781-43794. -- 2019. $//$ URL: $https://doi.org/10.1109/ACCESS.2019.2907725$ (Дата обращения: 13.03.2020)
	\bibitem{42}
	MD Jahedul Alam, Muhammad Ahsanul Habib. Mass Evacuation of Halifax, Canada: A Dynamic Traffic Microsimulation Modeling Approach. -- Procedia Computer Science. -- Volume 151. -- Pages 535-542. -- 2019. $//$ URL: $https://doi.org/10.1016/j.procs.2019.04.072$ (Дата обращения: 13.03.2020)
	\bibitem{43}
	Aimsun. $//$ URL: $https://www.aimsun.com/$ (Дата обращения: 17.03.2020)
	\bibitem{44}
	McTrans Center, University of Florida. TSIS-CORSIM. Overview. $//$ URL: $https://mctrans.ce.ufl.edu/mct/index.php/tsis-corsim/$ (Дата обращения: 17.03.2020)
	\bibitem{45}
	Caliper Corporation. TransModeler Traffic Simulation Software. $//$ URL: $https://www.caliper.com/transmodeler/default.htm$ (Дата обращения: 12.04.2020)
	\bibitem{46}
	Qiyuan Liu, Jian Sun, Ye Tian and Lu Xiong. Modeling and simulation of overtaking events by heterogeneous non-motorized vehicles on shared roadway segments. -- Simulation Modelling Practice and Theory. -- Volume 103. -- 2020. $//$ URL: $https://doi.org/10.1016/j.simpat.2020.102072$ (Дата обращения: 12.03.2020)
	\bibitem{47}
	Shuichao Zhang, Gang Ren and Renfa Yang. Simulation model of speed-density characteristics for mixed bicycle flow-Comparison between cellular automata model and gas dynamics model. -- Phisica A: Statistical Mechanics and its Applications. -- Volume 392. -- Issue 20. Pages 5110-5118. -- 2013. $//$ URL: $https://doi.org/10.1016/j.physa.2013.06.019$ (Дата обращения 17.03.2020)
	\bibitem{48}
	Sarosh I. Khan, Pawan Maini. Modeling Heterogeneous Traffic Flow. -- Transportation Research Record. -- Pages 234-241. -- 1999. $//$ URL: $https://doi.org/10.3141\%2F1678-28$ (Дата обращения: 14.03.2020)
	\bibitem{49}
	Jinxing Shen, Junje Qi, Feng Qiu and Changjang Zheng. Simulation of Road Capacity Considering the Influence of Buses. -- IEEE Access. -- Volume 7. -- Pages 144178-144187. -- 2019. $//$ URL: $https://doi.org/10.1109/ACCESS.2019.2942524$ (Дата обращения: 13.03.2020)
	\bibitem{50}
	Benjamin Seibold. Why a Traffic Flow Suddenly Turns Into a Traffic Jam. $//$ URL: $http://nautil.us/issue/71/flow/why-a-traffic-flow-suddenly-turns-into-a-traffic-jam$ (Дата обращения: 17.10.2019)
	\bibitem{51}
	Juan Calvo, Janjo Nieto and Mohamed Zagour. Kinetic model for vehicular traffic with continuum velocity and mean field interactions. -- Symmetry. -- Volume 11. -- Issue 9. -- 2019. $//$ URL: $https://doi.org/10.3390/sym11091093$ (Дата обращения: 13.03.2020)
	\bibitem{52}
	Потапова И.А., Бояршинова И.Н., Исмагилов Т.Р. Методы моделирования транспортного потока. --  Фундаментальные исследования. -- Номер 10-2. -- С.338-342. -- 2016. $//$ URL: $https://www.elibrary.ru/item.asp?id=27196298$ (Дата обращения: 13.04.2020)
	\bibitem{53}
	Крестов С.Г., Строганов Ю.В. Проверка времени исполнения сгенерированных запросов к графовой базе данных. – Новые информационные технологии в автоматизированных системах. -- Номер 20. -- С.235-238. -- 2017. $//$ URL: $https://elibrary.ru/item.asp?id=29109664$ (Дата обращения: 27.04.2019)
	\bibitem{54}
	Neo4j official documentation. Официальная документация Neo4j. $//$ URL: $https://neo4j.com/docs/operations-manual/current/tools/cypher-shell/$ (Дата обращения: 04.05.2019)
	\bibitem{55}
	Оселедец И.В. Прототипирование программных комплексов. -- Статья в сборнике трудов конференции «Научный сервис в сети Интернет: поиск новых решений». -- С.404-411. -- 2012. $//$ URL: $https://elibrary.ru/item.asp?id=22447021$ (Дата обращения: 30.04.2019)
	\bibitem{56}
	А.Волобуев. В России система платных дорог находится в стадии становления, а рентабельность сильно отстает от мировой. Такие выводы содержатся в недавно опубликованном исследовании KPMG об эффективности этого бизнеса в нашей стране и за рубежом. $//$ URL: $http://platniedorogi.ru/portfolio-view/the-profitability-of-russian-toll-roads$ (Дата обращения: 25.05.2020)
	\bibitem{57}
	Платные дороги в России. $//$ URL: $https://plusfinance.ru/platnye-dorogi-rossii.html$ (Дата обращения: 25.05.2020)
	\bibitem{58}
	Платные дороги займут 70 процентов России. $//$ URL: $https://motor.ru/news/no-free-roads-16-07-2019.htm$ (Дата обращения: 25.05.2020)
	\bibitem{59}
	Е.Астапенко. Платные трассы опустели: обслуживающие их компании хотят получить господдержку. $//$ URL: $https://www.kolesa.ru/news/platnye-trassy-opusteli-obsluzhivayushchie-ikh-kompanii-khotyat-poluchit-gospodderzhku$ (Дата обращения: 25.05.2020)
	\bibitem{60}
	Neo4j. Drivers \& Language Guides. $//$ URL: $https://neo4j.com/developer/language-guides/$ (Дата обращения: 26.05.2020)
	\bibitem{61}
	Estelle Scifo. Intoducing Neomap, a Neo4j Desktop application for spatial data. $//$ URL: $https://medium.com/neo4j/introducing-neomap-a-neo4j-desktop-application-for-spatial-data-3e14aad59db2$ (Дата обращения: 16.03.2020)
	\bibitem{62}
	Estelle Scifo. Visualizing shortest paths with neomap $\geq$ 0.4.0 and the Neo4j Graph Data Science plugin. $//$ URL: $https://medium.com/neo4j/visualizing-shortest-paths-with-neomap-0-4-0-and-the-neo4j-graph-data-science-plugin-18db92f680de$ (Дата обращения: 16.03.2020)
	\bibitem{63}
	Neo4j Cypher Refcard 4.0. $//$ URL: $https://neo4j.com/docs/cypher-refcard/current/$ (Дата обращения: 25.04.2020)
	\bibitem{64}
	PM4PY. Documentation. Petri Net properties. Creating a new Petri Net. $//$ URL: $https://pm4py.fit.fraunhofer.de/documentation$ (Дата обращения: 31.05.2020)
	\bibitem{65}
	А.Челушкин. Process mining: разработка алгоритмов для модификации библиотеки pm4py (для удобства аудита процессов). $//$ URL: $https://newtechaudit.ru/process-mining-pm4py/$
	\bibitem{66}
	Docs petri. $//$ URL: $https://petri.readthedocs.io/en/latest/$ (Дата обращения: 30.05.2020)
	\bibitem{67}
	petri 0.24.1. $//$ URL: $https://pypi.org/project/petri/$ (Дата обращения: 26.05.2020)
\end{thebibliography}
%\endgroup