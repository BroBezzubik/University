\documentclass[../main.tex]{subfiles}
\begin{document}
	Процесс нахождения решения в Прологе заключается в сопоставлении предиката цели с предикатами базы знаний. Этот процесс называется унификацией. Пусть Пролог-системе предъявлена цель parent(X, kristina), parent(Y, X). Пролог выделяет из нее первую подцель parent(X, kristina) и начинает сопоставлять ее с базой знаний (проводить унификацию). База знаний повторена ниже.
	
	\begin{lstlisting}[frame=single caption=Факты]
		parent(boris, alla).
		parent(bedros, filipp).
		parent(edmuntas, kristina).
		parent(alla, kristina).
		parent(kristina, denis).
	\end{lstlisting}
	
	Вначале сопоставляются первый предикат и подцель:
	parent(boris, alla) и parent(X, kristina)
	
	Первый аргумент boris сопоставляется с переменной X. Следует иметь в виду, что в Прологе различаются состояния переменных
	free (свободные) и bound (связанные). Если обе переменные связаны, то при унификации происходит их сравнение. Если одна из них свободна, то происходит присвоение. 
	Переприсвоени значений переменным не допускается. 
	Это существенно отличает Пролог от прочих языков. 
	Таким образом, переменной Х, которая пока является свободной, присваивается значение boris. 
	После этого унифицируются вторые аргументы, alla и kristina. Поскольку это константы, и alla не равно kristina, то унификация предиката parent(boris,alla) и подцели parent(X, kristina) заканчивается неудачей (fail).
	Поскольку в базе знаний несколько экземпляров предиката parent, такой предикат называется неоднозначным (non-derministic). Если предикат один, то он называется однозначным (deterministic). 
	В случае неоднозначного предиката после неудачи выполняется откат – переход к следующему экземпляру предиката. 
	При этом отменяется также присвоение значение переменным, если таковое имело место. Затем выполняется унификация предикатов parent(bedros, filipp) и parent(X, kristina)
	Очевидно, что результат унификации будет тот же, неудача (fail). 
	При откате на следующий предикат parent(edmuntas,kristina) картина будет иная: X 10 присвоится значение edmuntas, а сопоставление вторых аргументов будет также успешным, так как kristina = kristina. Таким образом, первая подцель окажется выполненной. 
	Пролог запоминает, какой экземпляр предиката сработал и устанавливает на следующий предикат указатель отката:
	\begin{lstlisting}
		parent(boris, alla).
		parent(bedros, filipp).
		parent(edmuntas, kristina).
		> parent(alla, kristina).
		parent(kristina, denis).
	\end{lstlisting}

	Итак, интерпретатор Пролога автоматически выполняет поиск
	решения. Механизм поиска реализован с помощью отката после неудачи.
	Откат происходит на следующий экземпляр неоднозначного предиката.
	Выполнение программы на Прологе (резолюция цели) заключается в
	унификации цели с базой знаний. 
\end{document}