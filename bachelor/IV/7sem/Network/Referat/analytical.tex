\section{Введение}
Что такое сеть? 
Это совокупность устройств и систем, которые подключены друг к другу (логически или физически) и общающихся между собой. 
Сюда можно отнести сервера, компьютеры, телефоны, маршрутизаторы и так далее.
Размер этой сети может достигать размера Интернета, а может состоять всего из двух устройств, соединенных между собой кабелем. 
Чтобы не было каши, разделим компоненты сети на группы:
\begin{enumerate}
	\item Оконечные узлы: Устройства, которые передают и/или принимают какиелибо данные. Это могут быть компьютеры, телефоны, сервера, какие-то
	терминалы или тонкие клиенты, телевизоры.
	\item Промежуточные устройства: Это устройства, которые соединяют оконечные узлы между собой. Сюда можно отнести коммутаторы, концентраторы, модемы, маршрутизаторы, точки доступа Wi-Fi.
	\item Сетевые среды: Это те среды, в которых происходит непосредственная
	передача данных. Сюда относятся кабели, сетевые карточки, различного
	рода коннекторы, воздушная среда передачи. Если это медный кабель, то
	передача данных осуществляется при помощи электрических сигналов. У
	оптоволоконных кабелей, при помощи световых импульсов. Ну и у беспроводных устройств, при помощи радиоволн.

\end{enumerate}

Для чего мы используем сети?
Первое что приходит на ум:

\begin{enumerate}
	\item Приложения: При помощи приложений отправляем разные данные между
	устройствами, открываем доступ к общим ресурсам. Это могут быть как
	консольные приложения, так и приложения с графическим интерфейсом.
	\item Хранилище: Используя сервер или рабочую станцию, подключенную к
	сети, создается хранилище доступное для других. Многие люди выкладывают туда свои файлы, видео, картинки и открывают общий доступ к ним
	для других пользователей. Пример, который на ходу приходит в голову,
	— это google диск, яндекс диск и тому подобные сервисы.
	\item Резервное копирование: Часто, в крупных компаниях, используют центральный сервер, куда все компьютеры копируют важные файлы для резервной копии. Это нужно для последующего восстановления данных, если оригинал удалился или повредился. Методов копирования огромное
	количество: с предварительным сжатием, кодированием и так далее.

\end{enumerate}

\newpage
\section{Приложения}
\subsection{Загрузчики}

Это файловые менеджеры, работающие по протоколу FTP, TFTP. Банальный пример — это скачивание фильма, музыки, картинок с файлообменников
или иных источников. К этой категории еще можно отнести резервное копирование, которое автоматически делает сервер каждую ночь. То есть это встроенные или сторонние программы и утилиты, которые выполняют копирование и
скачивание. Данный вид приложений не требует прямого человеческого
вмешательства. Достаточно указать место, куда сохранить и скачивание само
начнется и закончится


Скорость скачивания зависит от пропускной способности. Для данного
типа приложений это не совсем критично. Если, например, файл будет скачиваться не минуту, а 10, то тут только вопрос времени, и на целостности файла
это никак не скажется. Сложности могут возникнуть только когда нам надо за
пару часов сделать резервную копию системы, а из-за плохого канала и, соответственно, низкой пропускной способности, это занимает несколько дней.
Ниже приведены описания самых популярных протоколов данной группы:


FTP- это стандартный протокол передачи данных с установлением соединения. Работает по протоколу TCP (этот протокол в дальнейшем будет подробно рассмотрен). Стандартный номер порта 21. Чаще всего используется для
загрузки сайта на веб-хостинг и выгрузки его


TFTP- это упрощенная версия протокола FTP, которая работает без
установления соединения, по протоколу UDP. Применяется для загрузки образа
бездисковыми рабочими станциями. Особенно широко используется
устройствами Cisco для той же загрузки образа и резервных копий.

\subsection{Приложения в реальном времени}

Приложения, позволяющие передавать информацию в реальном времени.
Как раз к этой группе относится IP-телефония, системы потокового вещания,
видеоконференции. Самые чувствительные к задержкам и пропускной способности приложения. Представьте, что вы разговариваете по телефону и то, что
вы говорите, собеседник услышит через 2 секунды и наоборот, вы от собеседника с таким же интервалом. Такое общение еще и приведет к тому, что голоса
будут пропадать и разговор будет трудноразличимым, а в видеоконференция
превратится в кашу. В среднем, задержка не должна превышать 300 мс. К данной категории можно отнести Skype, Lync, Viber (когда совершаем звонок).

\section{Топология}

Топология делится на 2 большие категории: физическая и логическая.
Очень важно понимать их разницу. Итак, физическая топология — это как наша сеть выглядит. Где находятся узлы, какие сетевые промежуточные
устройства используются и где они стоят, какие сетевые кабели используются,
как они протянуты и в какой порт воткнуты. Логическая топология — это каким путем будут идти пакеты в нашей физической топологии. То есть физическая — это как мы расположили устройства, а логическая — это через какие
устройства будут проходить пакеты.

\subsection{Общаяя шина}

Одна из первых физических топологий. Суть состояла в том, что к одному длинному кабелю подсоединяли все устройства и организовывали локальную сеть. На концах кабеля требовались терминаторы. Как правило — это было
сопротивление на 50 Ом, которое использовалось для того, чтобы сигнал не отражался в кабеле. Преимущество ее было только в простоте установки. С точки
зрения работоспособности была крайне не устойчивой. Если где-то в кабеле
происходил разрыв, то вся сеть оставалась парализованной, до замены кабеля.

\subsection {Кольцевая топология}

В данной топологии каждое устройство подключается к 2-ум соседним.
Создавая, таким образом, кольцо. Здесь логика такова, что с одного конца
компьютер только принимает, а с другого только отправляет. То есть, получается передача по кольцу и следующий компьютер играет роль ретранслятора
сигнала. За счет этого нужда в терминаторах отпала. Соответственно, если гдето кабель повреждался, кольцо размыкалось и сеть становилась не работоспособной. Для повышения отказоустойчивости, применяют двойное кольцо, то
есть в каждое устройство приходит два кабеля, а не один. Соответственно, при
отказе одного кабеля, остается работать резервный

\subsection{Звезда}

Все устройства подключаются к центральному узлу, который уже является ретранслятором. В наше время данная модель используется в локальных сетях, когда к одному коммутатору подключаются несколько устройств, и он является посредником в передаче. Здесь отказоустойчивость значительно выше,
чем в предыдущих двух. При обрыве, какого либо кабеля, выпадает из сети
только одно устройство. Все остальные продолжают спокойно работать. Однако если откажет центральное звено, сеть станет неработоспособной

\subsection{Полносвязная}

Все устройства связаны напрямую друг с другом. То есть с каждого на
каждый. Данная модель является, пожалуй, самой отказоустойчивой, так как не
зависит от других. Но строить сети на такой модели сложно и дорого. Так как в
сети, в которой минимум 1000 компьютеров, придется подключать 1000 кабелей на каждый компьютер.

\section{Сетевые модели}

На этапе зарождения компьютеров, у сетей не было единых стандартов.
Каждый вендор использовал свои проприетарные решения, которые не работали с технологиями других вендоров. Конечно, оставлять так было нельзя и нужно было придумывать общее решение. Эту задачу взвалила на себя международная организация по стандартизации (ISO — International Organization for
Standartization). Они изучали многие, применяемые на то время, модели и в
результате придумали модель OSI, релиз которой состоялся в 1984 году. Проблема ее была только в том, что ее разрабатывали около 7 лет. Пока специалисты спорили, как ее лучше сделать, другие модели модернизировались и набирали обороты. В настоящее время модель OSI не используют. Она применяется
только в качестве обучения сетям. Мое личное мнение, что модель OSI должен
знать каждый уважающий себя админ как таблицу умножения. Хоть ее и не
применяют в том виде, в каком она есть, принципы работы у всех моделей
схожи с ней.

\begin{center}
	\begin{tabular}{|c|}
		\hline
		Прикладной уровень \\
		\hline
		Уровень представления \\
		\hline
		Сеансовый уровень \\
		\hline
		Транспортный уровень \\
		\hline
		Сетевой уровень \\
		\hline
		Канальный уровень \\
		\hline
		Канальный уровень \\
		\hline
		Физический уровень \\
		\hline
	\end{tabular}
\end{center}

\begin{enumerate}
	\item Физический уровень (Physical Layer): определяет метод передачи данных, какая среда используется (передача электрических сигналов, световых импульсов или радиоэфир), уровень напряжения, метод кодирования
	двоичных сигналов.

	\item Канальный уровень (Data Link Layer): он берет на себя задачу адресации в пределах локальной сети, обнаруживает ошибки, проверяет целостность данных. Если слышали про MAC-адреса и протокол «Ethernet», то
	они располагаются на этом уровне.
	\item Сетевой уровень (Network Layer): этот уровень берет на себя объединения участков сети и выбор оптимального пути (т.е. маршрутизация). Каждое сетевое устройство должно иметь уникальный сетевой адрес в сети.
	Думаю, многие слышали про протоколы IPv4 и IPv6. Эти протоколы работают на данном уровне.
	\item Транспортный уровень (Transport Layer): Этот уровень берет на себя
	функцию транспорта. К примеру, когда вы скачиваете файл с Интернета,
	файл в виде сегментов отправляется на Ваш компьютер. Также здесь вводятся понятия портов, которые нужны для указания назначения к конкретной службе. На этом уровне работают протоколы TCP (с установлением соединения) и UDP (без установления соединения).
	\item Сеансовый уровень (Session Layer): Роль этого уровня в установлении,
	управлении и разрыве соединения между двумя хостами. К примеру,
	когда открываете страницу на веб-сервере, то Вы не единственный посетитель на нем. И вот для того, чтобы поддерживать сеансы со всеми пользователями, нужен сеансовый уровень.
	\item Уровень представления (Presentation Layer): Он структурирует
	информацию в читабельный вид для прикладного уровня. Например,
	многие компьютеры используют таблицу кодировки ASCII для вывода
	текстовой информации или формат jpeg для вывода графического изображения.

	\item Прикладной уровень (Application Layer): Наверное, это самый понятный для всех уровень. Как раз на этом уроне работают привычные для
	нас приложения — e-mail, браузеры по протоколу HTTP, FTP и остальное.
\end{enumerate}

\section{Виды сетей}

В целом сети можно разбить на 3 группы:
\begin{enumerate}
	\item Локальная сеть;
	\item Региональная;
	\item Глобальная.
\end{enumerate}
	
	Локальная вычислительная сеть (Local Area Network, LAN) — компьютерная сеть, покрывающая обычно относительно небольшую территорию или небольшую группу зданий (дом, офис, фирму, институт). Локальная сеть строится с помощью различных топологий (см. в разделе топологии).
	
	
	Региональная (городская) сеть (Metropolitan Area Network, MAN) – сеть, соединяющая множество локальных сетей в рамках одного района, города или региона. Самым простым примером городской сети является система кабельного телевидения.
	
	
	Глобальная сеть — это объединение компьютеров, расположенных на большом расстоянии, для общего использования мировых информационных ресурсов. В настоящее время для обеспечения связи в глобальных сетях выработаны единые правила — технология Интернет.
	
	Интернет (англ. Internet) — всемирная система объединённых компьютерных сетей, построенная на использовании протокола IP и маршрутизации пакетов данных. Интернет образует глобальное информационное пространство, служит физической основой для Всемирной паутины и множества других систем (протоколов) передачи данных. Часто упоминается как «Всемирная сеть» и «Глобальная сеть».