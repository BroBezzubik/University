\section{Теоритическая часть}

\subsection{Распределения}

\subsubsection{Равномерное распределение}

Случайная величина имеет непрерывное равномерное распределение на отрезке $[a,b]$, где $a,b \in R $, если ее плотность $f_x(X)$ имеет вид:
\begin{equation}
f_x(x)=\begin{cases}
\frac{1}{b-a} \quad x \in [a,b] \\
0 \quad x \notin [a,b].
\end{cases}
\end{equation}

Интегрируя определенную выше плотность получаем:
\begin{equation}
F_x(x) \equiv P(X <= x) = \begin{cases}
0, \quad x < a \\
\frac{x - a}{b - a}, \quad a <= x < b \\
1, \quad x >= b
\end{cases}
\end{equation}

\subsubsection{Распределение Пауссона}

Распределение Паусона используется для моделирования количетсва событий проиходящих в заданном временном интервале.


\begin{equation}
f(x; \lambda) = \frac{e^{-\lambda * \lambda^x}}{!x} \quad x = 0, 1, 2, 3
\end{equation}

Функция плотности имеет вид:

\begin{equation}
F(x; \lambda) = \sum_{i=0}^{x}\frac{e^{-\lambda * \lambda^i}}{i!} 
\end{equation}

$\lambda$ - параметр формы, который указывает среднее количество событий в данном временном интервале.

\subsection{Протяжка модельного времени}
Основная функция протягиваня модельного времени состоит в реализации алгоритма функционирования.
Иматация взаимодействий отдельных устройств системы происходит с помощью управляющей программы.

Управляющая программа реализуется в основном по двум принципам:
\begin{enumerate}
	\item принцип $\triangle$t;
	\item событийный принцип.
\end{enumerate} 

Так же можно применять комбинирующий метод, сочетающий в себе два указанных принципа

\subsubsection{Метод $\triangle$t}

Принцип $\triangle$t заключается в последовательном анализе состояний всех блоков в момент $t + \triangle t$ по заданному состоянию блоков в момент t.
При этот новое состояние блоков определяется в соответсвии с их алогоритмическим описением с учетом действующих случайных факторов, задаваемых распределениями вероятности.
В результате такого анализа принимается решение о том, какие общесистемные события должны имитироваться программной моделью на данный момент времени.

Достоинством данного метода является равномерность протягивания модельного времени.

Основной недостаток этого принципа заключается в значительных затратах машинного времени на реализацию моделирования системы. 
При недостаточно малом $\triangle$t появляется опасность пропуска отдельных событий в системе, что исключает возможность получения адекватных резульатов при моделировании.

\subsubsection{Событийный метод }

Характерным свойством систем обработки информации является то, что состояние отдельных устройств изменяется в дискретные моменты времени, совпадающие с моментами времени поступления сообщений в систему, временем поступления окончания задачи, времени поступления аварийных сигналов и т.д. 
Поэтому моделирование и продвижение времени можно проводить с использованием событийного принципа. 
При его использовании состояние всех блоков системы анализируется лишь в момент появления или наступления какого-либо события. 
Момент поступления следующего события определяется минимальным значением из списка будущих событий, представляющего собой совокупность моментов ближайшего изменения состояния каждого из блоков системы

Любое событе в этом подходе можно описать и использованием пяти осей:

\begin{enumerate}
	\item 1 - момент появления события от источника ниформации;
	\item 2 - момент освобождения обсуживающего аппарата (ОА);
	\item 3 - моменты сбора статистики;
	\item 4 - время окончания моделирования;
	\item 5 - текущее время.
\end{enumerate}

С помощью этих осей задаются интервалы обслуживания сообщений и соответствующие моменты.
На основе этих данных формируется список будущих событий (СБС).

В общем виде метод можно описать след. образом:
\begin{enumerate}
	\item для блоков активных блоков заводят свой элемент в СБС;
	\item в СБС заносят время ближайшего события от любого активного блока;
	\item становится активным программный имитатор источника событий и производит псевдослучайную величину, которая определяет момент появления первого сообщения от источника сообщений и которую помещают в СБС;
	\item становится активным имитатор для выроботки величины для ОА, которую тоже необходимо занести в СБС;
	\item в момент первого сбора стат. определяется стандартный шаг сбора стат., который заносится в СБС;
	\item выполняется протяжка времени.
\end{enumerate}

Задача блока сбора статистики заключается в накоплении численных значений, которые необходимы для вычисления заданных параметров моделируемой системы. Как правило, при моделировании СМО к таким значениям относят:

\begin{enumerate}
	\item среднее время; 
	\item среднее знач длины очеред; 
	\item коэффицент загрузки ОА
	\item вероятность потери сообщений.
\end{enumerate}