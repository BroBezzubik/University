\section{Введение}
FTP-клиент  FTP File Transfer Protocol — Компьютерная программа для упрощения доступа к FTP серверу. 
В зависимости от назначения может либо предоставлять пользователю простой доступ к удаленному FTP-серверу в режиме текстовой консоли, беря на себя только работу по пересылке команд пользователя и файлов, либо отображать файлы на удаленном сервере, как если бы они являлись частью файловой системы компьютера пользователя, либо и то и другое. 
В последних двух случаях FTP-клиент берёт на себя задачу интерпретации действий пользователя в команды протокола FTP, тем самым давая возможность использовать протокол передачи файлов без ознакомления со всеми его премудростями.

Частными примерами использования FTP-клиента могут быть:
\begin{enumerate}
	\item публикация страниц на сайте Веб-разработчика;
	\item скачивание музыки, программ и любых других файлов данных обычным пользователем интернета;
\end{enumerate}

В простейшем для пользователя (но при этом наиболее комплексном) случае FTP-клиент представляет собой эмулятор файловой системы, которая просто находится на другом компьютере. 
С этой файловой системой можно совершать все привычные пользователю действия: копировать файлы с сервера и на сервер, удалять файлы, создавать новые файлы. 
В отдельных случаях возможно также открытие файлов — для просмотра, запуска программ, редактирования. 
Необходимо учитывать лишь, что открытие файла подразумевает его предварительное скачивание на компьютер пользователя. Примерами таких программ могут служить:
\begin{enumerate}
	\item интернет-браузеры (часто работают в режиме «только чтение», то есть не позволяют добавлять файлы на сервер);
	\item онлайн клиенты, работа с которыми осуществляется посредством любого интернет-браузера;
\end{enumerate}

В данной работе реализуется FTP клиент для Linux. Пользователь может подключиться к удаленному серверу используя ранее созданный профиль в удаленной системе. 
Анонимный пользователь не допускается.
\begin{enumerate}
	\item получения файлов с сервера компании;
	\item в курсе базового ознакомления с консолью;
	\item обучение в рамках курса “Сети"
\end{enumerate}

Целью проекта является получения программы, позволяющей пользователю подключаться к удаленному серверу и отправлять, загружать, манипулировать данными. 
Для достижения поставленной цели необходимо выполнить следующие поставленные задачи:
\begin{enumerate}
	\item аналитические:
	\begin{enumerate}
		\item формализовать предметную область;
		\item выбрать объект исследования;
		\item обосновать выбор;
	\end{enumerate}
	\item конструкторские:
	\begin{enumerate}
		\item описать алгоритмы;
		\item указать собенности приктической реализации;
		\item формализовать описанные данные, ограничения и допущения, требования к программному обеспечению и способов взаимодействия программы с пользователем;
	\end{enumerate}
	\item технологические:
	\begin{enumerate}
		\item обосновать выбор программной реализации;
		\item описать требуемое программное обеспечение;
		\item описать входные и выходные данные разработанного программного продукта;
		\item тестирование;
	\end{enumerate}
\end{enumerate}