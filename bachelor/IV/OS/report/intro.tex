\newpage
\anonsection{Введение}
На сегодняшний день существует множество причин по которым клавиатура должна быть сконфигурирована для использования в качестве мыши.
Одной из них является распространенное использование беспроводных компьютерных мышей с батарейным питанием, которые любят иссекать свой запас в самый неподходящий момент, да и могут просто выйти из строя.
Так же немаловажным является, то что людям с проблемой мобильности рук намного проще нажимать клавиши на клавиатуре, чем двигать рукой по столу.

Существует несколько путей достижения данной функциональности в операционой системе (ОС) Linux, но достаточно часто они очень сложны для рядового пользователя или лишены некоторых ключевых функций.

Среди существующих аналогов можно выделить следующие:
\begin{enumerate}
	\item xbindkeys - работает, но настройка окажется сложной для рядового пользователя;
	\item MouseKeys - поддерживает только Ubuntu и требует наличие цифровой клавиатуры.
\end{enumerate}

Практической целью курсовой работы является разработка загружаемого модуля ядра, который позволит пользователю легко управлять курсором мышь с помощью клавиатуры и настраивать конфигурацию.

В соответствии с заданием на курсовой проект необходимо разработать модуль ядра для управления курсором мыши с помощью клавиатуры.

Для решения поставленной цели необходимо выполнить следующие задачи:
\begin{enumerate}
	\item проанализировать процесс обработки прерываний;
	\item проанализировать процедуры доступа к портам ввода/вывода;
	\item проанализировать драйвер символьных устройств;
	\item проанализировать подсистему ввода ядра;
	\item разработать модуль ядра для управления курсором мыши.
\end{enumerate}

