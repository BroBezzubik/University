\section{Аналитичесий раздел}

\subsection{Распределение}

\subsubsection{Равномерное распределение}

Случайная величина имеет непрерывное равномерное распределение на отрезке $[a,b]$, где $a,b \in R $, если ее плотность $f_x(X)$ имеет вид:
\begin{equation}
f_x(x)=\begin{cases}
\frac{1}{b-a} \quad x \in [a,b] \\
0 \quad x \notin [a,b].
\end{cases}
\end{equation}

Интегрируя определенную выше плотность получаем:
\begin{equation}
F_x(x) \equiv P(X <= x) = \begin{cases}
	0, \quad x < a \\
	\frac{x - a}{b - a}, \quad a <= x < b \\
	1, \quad x >= b
\end{cases}
\end{equation}

\subsubsection{Распределение Пауссона}

Распределение Паусона используется для моделирования количетсва событий проиходящих в заданном временном интервале.


\begin{equation}
f(x; \lambda) = \frac{e^{-\lambda * \lambda^x}}{!x} \quad x = 0, 1, 2, 3
\end{equation}

Функция плотности имеет вид:

\begin{equation}
	F(x; \lambda) = \sum_{i=0}^{x}\frac{e^{-\lambda * \lambda^i}}{i!} 
\end{equation}

$\lambda$ - параметр формы, который указывает среднее количество событий в данном временном интервале.

