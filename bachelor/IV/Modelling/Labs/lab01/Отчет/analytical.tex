\section{Аналитический раздел}

В данном разделе поставлена цель работы и указаны задачи необходимые для выполнения данной. Описаны способы полунчения псевдослучайных чисел. 
Разобран метод BBS.
Выбран критерий случайности

\subsection{Цель работы}
Реализовать критерий оценки случайности последовательности. 
Сравнить результаты работы данного критерия на одноразрядных, двухразрядных и трехразрядных последовательностях псевдослучайных целых чисел.
Последовательности получать алгоритмическим способом, табличным способом и путём ручного ввода.

Для выполнения выше поставленной цели необходимо выполнить следующие:
\begin{enumerate}
	\item Ответить на вопросы:
		\begin{enumerate}
			\item Что такое псевдослучайные числа
			\item Способы получения
		\end{enumerate}
	\item Выбрать и реализовать алгоритмический метод получения псевдослучайных целых чисел.
	\item Выбрать и реализовать критерий для оценки случайности
\end{enumerate}

\subsection{Псевдослучайные числа}
Данные числа называют псевдослучайными послько даже пройдя все статистические испытания на случайность и равномерность распределения остаются полностью детерминированными. То есть, если каждый цикл работы генератора начинается с теми же условиями, то на выходе мы получим одни и те же полседовательности.

\subsection{Способы получения}
На практике используются 3 основных способа:
\begin{enumerate}
	\item Аппартный
	\item Табличная схема
	\item Алгоритмический способ
\end{enumerate}

Каждый из данных способов облдает своими достоинствами и недостатками:
\begin{enumerate}
	\item Аппаратный \begin{enumerate}
		\item Достоинтсва:\begin{enumerate}
			\item Запас чисел неограничен;
			\item Расходуется мало операций;
			\item Не занимет место в ОП.
		\end{enumerate}
		\item Недостатки: \begin{enumerate}
			\item Требуется переодическая проверка на случайность;
			\item Нельзя воспроизводить последовательность;
			\item Используются спец устройства. Надо стабилизировать;
		\end{enumerate}
	\end{enumerate}
	\item Табличный \begin{enumerate}
		\item Достоинтсва: \begin{enumerate}
			\item Требуется однократная проверка;
			\item Можно воспроизводить последовательность;
		\end{enumerate}
		\item Недостатки:\begin{enumerate}
			\item Запас чисел ограничен;
			\item Занимает место в оперативной памяти и требует время на обращение;
		\end{enumerate}
	\end{enumerate}
	\item Алгоритмический \begin{enumerate}
		\item Достоинтсва: \begin{enumerate}
			\item Одна проверка;
			\item Многократное воспроизведение;
			\item Относительное малое место в ОП;
			\item Не использует внешнее устройство;
		\end{enumerate}
		\item Недостатки: \begin{enumerate}
			\item Запас чисел ограничен ее периодом;
			\item Требуются затраты машинного времени;
		\end{enumerate}
	\end{enumerate}
\end{enumerate}

\subsection{BlumBlumShub}
Широкое распространение получил алгоритм генерации псевдослучайных чисел, называемый алгоритмом BBS (от фамилий авторов — L. Blum, M. Blum, M. Shub) или генератором с квадратичным остатком. 
Для целей криптографии этот метод предложен в 1986 году.
Он заключается в следующем. Вначале выбираются два больших простых1 числа p и q. 
Числа p и q должны быть оба сравнимы с 3 по модулю 4, то есть при делении p и q на 4 должен получаться одинаковый остаток 3. 
Далее вычисляется число M = p* q, называемое целым числом Блюма. Затем выбирается другое случайное целое число х, взаимно простое (то есть не имеющее общих делителей, кроме единицы) с М. Вычисляем $ x_0 = x^2 mod M$. $x_0$ называют стартовым числом генератора.

На каждом n-м шаге работы генератора вычисляется $x_{n+1} = x_n^2 mod M$. Результатом n-го шага является один (обычно младший) бит числа $x_{n + 1}$
Иногда в качестве результата принимают бит чётности, то есть количество единиц в двоичном представлении элемента.

\subsection{Критерий сериальной корреляции}
Можно подсчитать следующую статистику:


$C = \frac{n * (U_0 * U_1 + U_1 * U_2 + ... + U_{n-2} * U{n - 1} + U_{n - 1} * U_0) - (U_0 + U_1 + ... + U_{n-1})^2}
{n * (U_0^2 + U_1^2 + ... + U_{n-1}^2) - (U_0 + U_1 + ... + U_{n-1})^2}$


Это коэффициенты сериальной корреляции, мера зависимости $ U_{j+1} $ от $ U_j $. Коэффициент корреляции всегда лежит между -1 и 1. Когда он равен 0 или очень мал, значит величины $U_{j+1}$ и $U_j$ независимы одна от другой (между ними нет линейной зависимости); если же значение коэффициента корреляции равно +1 или -1, это означает полную линейную зависимость.