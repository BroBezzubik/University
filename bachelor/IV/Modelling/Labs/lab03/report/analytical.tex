\newpage
\section{Аналитический раздел}
В данном разделе указана цель работы. Представлено уравнение Колмогорова в общем виде

\subsection{Цель работы}
Формализовать систему, количество состояний которой вводится пользова-телем. Нужно найти среднее, относительное время нахождения системы в каж-дом из её состояний. Система формализуется матрицей, в заголовках строк и столбцов которой находятся номера состояний: $S_1, S_2, ..., S_n$. 
На пересечениях стоят интенсивности перехода из состояния в состояние. 
Необходимо найти среднее относительное время нахождения системы в каждом из её состояний.

\subsection{Уравнение Колмогорова в общем виде}
\begin{equation}
\frac{dp_i}{dt} = \sum_{j - 1}^{n}p_j(t)\lambda_{ji} - p_i(t)\sum_{j - 1}^{n} \lambda_o \quad i=1,...,n
\end{equation}

Выше, учитывается, что для состояний не имеющих непосредственных перехо-дов, можно считать $\lambda_0 - \lambda_{ji} = 0$

Имея в распоряжении размеченный граф состояний, можно найти все вероятности состояний $p_i(t)$ как функции времени. 
Для этого составляются и решаются так называемые уравнения Колмогорова особого вида дифференциальные уравнения, в которых неизвестными функциями являются вероятности состояний.

Общее правило составления уравнений Колмогорова: в левой части каждого из них стоит производная вероятности какого-то (i-го) состояния. 
В правой части сумма произведений вероятностей всех состояний, из которых идут стрелки в данное состояние, на интенсивности соответствующих потоков событий, минус суммарная интенсивность всех потоков, выводящих систему из данного состоя-ния, умноженная на вероятность данного (i-го) состояния.

При $t->\inf$вероятности состояний будут стремиться к пределам, т.к. в теории случайных процессов доказывается, что если число состояний системы конечно и из каждого из них можно (за конечное число шагов) перейти в любое другое, то финальные вероятности существуют, которые, если существуют и не зависят от начального состояния системы, называются финальными вероятностями состояний.

При $t->\inf$ в системе $S$ устанавливается предельный стационарный режим, в ходе которого система случайным образом меняет свои состояния, но их вероятности уже не зависят от времени. Финальную вероятность состояния $S_i$ можно истолковать как среднее относительное время пребывания системы в этом состоянии.