\documentclass[a4paper, 14pt]{article}
\usepackage[utf8]{inputenc}
\usepackage[russian]{babel}
\usepackage{graphicx}
\usepackage{float}
\usepackage{amssymb}
\usepackage{amsmath}
\usepackage{pgfplots}
\usepackage{hyperref}
\usepackage{subfiles}
\usepackage[shortlabels]{enumitem}
\usepackage{titlesec}
\usepackage{algorithm}
\usepackage{algpseudocode}


\titleformat*{\section}{\LARGE\bfseries}
\titleformat*{\subsection}{\Large\bfseries}
\titleformat*{\subsubsection}{\large\bfseries}
\titleformat*{\paragraph}{\large\bfseries}
\titleformat*{\subparagraph}{\large\bfseries}


\usepackage{biblatex}[
backend = biber,
style=alphabetic,
sorting=ynt]
\addbibresource{./library/document}

% Настройка Листинга
\usepackage{listings}
\lstset{
	language=Python,
	numbers=left,
	frame=single,
	breaklines=true,
	breakatwhitespace=true,
	title=lstname,
	tabsize=2,
	framexrightmargin=20mm
}

\DeclareUnicodeCharacter{03BB}{\ensuremath{\lambda}}
\DeclareUnicodeCharacter{03B7}{\ensuremath{\eta}}
\DeclareUnicodeCharacter{03C4}{\ensuremath{\tau}}
\DeclareUnicodeCharacter{03C1}{\ensuremath{\rho}}

\begin{document}
	\begin{titlepage}
		\begin{center}
			\begin{LARGE}
				Лабараторная работа №1
			\end{LARGE}
			
			\begin{Large}
				\vspace{5cm}
				Тема: Программная реализация приближенного аналитического метода и простейших численных алгоритмов  первого порядка точности при решении  задачи Коши для ОДУ.
				
				\vspace{5cm}
				Студент: Барсуков Н.М.
				
				Группа: ИУ7-66Б
				
				Оценка (баллы):
				
				Преподаватель: Градов В.М.
			\end{Large}
		\end{center}
	\end{titlepage}

	\newpage
	\tableofcontents
	
	\newpage
	\section*{Введение}
	
	\newpage
	\section{Аналитическая часть}
	
	\newpage
	\section{Листинг}
	
	\newpage
	\section{Результат работы программы}
	
	\newpage
	\section{Ответы на вопросы}
	
	\newpage
	\section{Заключение}
	
	\newpage
	\section{Список литературы}
	

\end{document}
