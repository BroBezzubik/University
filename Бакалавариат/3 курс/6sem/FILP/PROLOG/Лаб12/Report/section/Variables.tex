\documentclass[../main.tex]{subfile}
\begin{document}
	В прологе как почти и во всех других языках программирования есть понятие переменной. 
	Различие заключается в том что в прологе понятие переменной ближе к математическому смыслу.
	parent(Parent, child) - Parent является переменной. Которое является не конкретным элементом, а элементом множества parent (может быть пустым).
	Переменные имеет локальную видимость внутри фактов и правил

	В прологе различается двоя состояния переменной:
	\begin{enumerate}[1)]
		\item free (Свободные)
		\item bound (связанные)
	\end{enumerate}
\end{document}