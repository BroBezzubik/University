\documentclass[12pt,a4paper]{scrartcl}
\usepackage[utf8]{inputenc}
\usepackage[english,russian]{babel}
\usepackage{misccorr}
\usepackage{graphicx}
\usepackage{amsmath}
\usepackage{amsfonts}
\usepackage{verbatim}
\usepackage{listings}
\usepackage{pgfplots}
\usepackage{graphicx}
\usepackage{pdfpages}
\usepackage{multirow}

\DeclareUnicodeCharacter{03BC}{\ensuremath{\mu}}
\DeclareUnicodeCharacter{03B3}{\ensuremath{\gamma}}
\DeclareUnicodeCharacter{03C3}{\ensuremath{\sigma}}
\DeclareUnicodeCharacter{03B8}{\ensuremath{\theta}}
\DeclareUnicodeCharacter{03C7}{\ensuremath{\chi}}
\DeclareUnicodeCharacter{03B1}{\ensuremath{\alpha}}

\usepackage{algorithm}
\usepackage{algpseudocode}

\begin{document}
	\section{Лабораторная N1}
	\begin{enumerate}
		\item Укажите интервалы значений аргумента, в которых можно считать решением заданного уравнения каждое из первых 4-х  приближений Пикара. Точность результата оценивать до второй цифры после запятой. Объяснить свой ответ.
		\begin{enumerate}
			\item 1ое приближение: [0 - 0.9] Значение функции совпадает c приближением на данном интервале до 2ух знаков. 
			\item 2ое приближение: [0 - 1.18] Значение функции совпадает c приближением на данном интервале до 2ух знаков. 
			\item 3ье приближение: [0 - 1.36] Значение функции совпадает c приближением на данном интервале до 2ух знаков.
			\item 4ое приближение: [0 - 1.51] Значение функции совпадает c приближением на данном интервале до 2ух знаков.
		\end{enumerate}
		\item Пояснить, каким образом можно доказать правильность полученного результата при фиксированном значении аргумента  в численных методах. 
		\begin{enumerate}
			\item Численные методы зависят от шага, по этому мы сравниваем значение полученные Эйлером при разных его величинах. На пример: берем шаг равный 0.00001 получаем (для примера) 200, а при 0.0000001 получаем 300. Видим, что слишком большая разница которая нас не устраивает, продолжаем уменьшать шаг. И так до тех пора пока не получаем устраивающую нас точность. Правда есть вероятно того, что мы выйдем за разрядность и будет происходить округление.
		\end{enumerate}
		\item Из каких соображений выбирался корень уравнения в неявном методе?
		\begin{enumerate}
			\item Выбирается наименьший из двух корней для сохранения непрерывности функции
		\end{enumerate}
		\item Каково значение функции при x=2, т.е. привести значение u(2).		
		\begin{enumerate}
			\item $ u(2) = 316.843976$ При уменьшении шага, погрешность уменьшится.
		\end{enumerate}
	\end{enumerate}
	
	\section{Лабораторная N2}
	\begin{enumerate}
		\item Вопрос: Какие способы тестирования программы можно предложить?\begin{enumerate}
			\item 1) Приравнять друг к другу знание Rk и Rp к 0. Потери в контуре будут отсутствовать. На графики будем видеть незатухающую синусоиду
			2) Сравнить результаты методов, с малым шагом
		\end{enumerate}
		\item Вопрос: Получите систему разностных уравнений для решения сформулированной задачи неявным методом трапеций. Опишите
		алгоритм реализации полученных уравнений. \begin{enumerate}			
			\item Ответ:\begin{equation}
			\text{Возьмем систему:}
			\begin{cases}
			\frac{dl}{dt} & =\frac{U - (R_k + R_p(I))I}{L_k}\\
			\frac{dU}{dy} & =-\frac{I}{C_k}
			\end{cases}
			\text{Запишем выражение:}
			\end{equation}
			
			Запишем выражение:
			
			\begin{equation}
			I_{n+1} = I_n + \Delta t\frac{f(I_n, U_{n+1})}{2}
			\end{equation}
			\begin{equation}
			U_{n+1} = U_n + \Delta t\frac{g(l_n) + g(l_{n+1})}{2}
			\end{equation}
			
			Подставим выражение f и g:
			
			\begin{equation}
			l_{n + 1} = l_n + \Delta t\frac{U_n - (R_k + R_p(I_n))I_n + U_{n+1} - R_k + Rp(l_{n+1}))I_{n+1}}{2L_k}
			\end{equation}
			\begin{equation}
			U_{n+1} = U_n - \Delta\frac{l_n + l_{n+1}}{2C_k}
			\end{equation}
			
			Подставив $U_{n+1}$ из второго уравнения в первое, решим его относительно $l_{n+1}$
			
			\begin{equation}
			l_{n+1} = \frac{-2C_kR_p(I_n)I_n\Delta t + 4C_kI_kI_n - 2C_kI_nR_k\Delta t + 4C_kU_n\Delta t - I_n\Delta t^2}{den}
			\end{equation}
			
			Данное уравнение решается методом простых итераций $x^{(s)} = f(x^(s-1))$
			Получив $l_{n+1}$ подставим его значение в уравнение $U_{n+1} = U_n - \Delta t \frac{l_n + l_{n + 1}}{2C_k}$
			
		\end{enumerate}
		\item Вопрос: Из каких соображений проводится выбор того или иного метода,
		учитывая, что чем выше порядок точности метода, тем он более сложен? \begin{enumerate}
			\item Ответ: Для метод четвёртого порядка точности должны быть четвёртые производные ограниченные, то есть правая часть должна быть непрерывна и ограничена вместе со своими четвёртыми производными. Если это не так, то метод четвёртого порядка точности не обеспечивает этот порядок.
			
			Для метода второго порядка правая часть должна быть непрерывна и ограничена вместе со своими производными до второго порядка. Если это не так, то метод второго порядка точности не обеспечивают этот порядок и следует использовать метод Эйлера.
			
		\end{enumerate}
	\end{enumerate}
	
	\section{Лабораторная работа N3}
	\begin{enumerate}
		\item Какие способы тестиования программы можно предложить?  \begin{enumerate}
			\item При $F_0 = 0 T(x) = T_0 +- \epsilon$
			\item Должна быть положительная производная функции T(x) при $F_0 < 0$
			\item При отрицательном радиусе стержня $R < 0$, должны наблюдаться гормонические колебания
			
		\end{enumerate}
		\item Получите простейший разностный аналог нелинейного краевого условия \begin{enumerate}
			\item $
			x = l, -k(l)\frac{dt}{dx} = \alpha_N(T(l) - T_0) + \phi(T) $

			Разностная аппроксимация краевого условия:
			
			$\frac{Y_{N-1} - Y_N}{h} * k_N = \alpha_N(y_N - T_0) + \phi(Y_N)$
			
		\end{enumerate}
		\item Опишите алгоритм применения метода прогонки, если при х = 0 краевое условие линейное (как в настоящей работе), а при x = l, как в п.2 \begin{enumerate}
			\item Используя простейшую аппроксимацию первых производных односторенними разностями, получим:
			
			$\xi_1 = 1, \nu = \frac{F_0*h}{k_0}$
			
			Далее, найдем, прогоночные коэффициенты:
			
			$\xi_{n+1} = \frac{C_n}{B_n - A_n * \xi_n}, \nu_{n+1} = \frac{F_n + A_n * \nu_n}{B_n - A_n * \Xi_n}$
			
			Учитывая, что $y_{n-1} = \xi_n*y_n + \nu_n$ найдем:
			
			$y_N = \frac{k_N * \nu_N + h*\alpha * \beta - h * \phi(y_N)}{k_N * (1 - \xi_N) + h\alpha}$
			
		\end{enumerate}
		\item Опишите алгоритм определения единствнного значения сеточной функции $y_p$ в одной заданной точке p. Использовать встречную прогонку, т.е. 
		комбинацию правой и левой прогонок (Лекция 8). Краевые условия линейные \begin{enumerate}
			\item Пусть: $\eta_1 = \frac{F_0}{B_0}; \xi = \frac{C_0}{B_0}; \eta_N = \frac{A_N}{B_N}; \xi_N = \frac{F_N}{B_N}$
			
			Прямой ход $ (1 <= i <= p - 1)$:
			
			$\xi_{i+1} = \frac{C_i}{B_i - A_i * \xi_i}; \eta_{i+1} = \frac{F_i + A_i * \eta_i}{B_i - A_i * \xi_i}$
			
			Обратный ход: ($p <= i <= N-1$):
			
			$\xi_i^- = \frac{A_i}{B_i - C_i * \xi_{i+1}}; \eta_i^- = \frac{F_i + C_i * \eta_i + 1}{B_i - C_i * \xi_i^- + 1}$
			
			\vspace{0.5cm}
			
			$ y_{p-1} = \xi_p * y_p + \eta_p \\ 
			y_p+1 = \xi_p^- * y_p + \eta^- \\
			A_p * y_{p-1} * B_p * y_p + C_p * y_{p+1} = -P_p \} => y_p = \frac{F_p + A_p * \eta_p + C_p * \eta_p^- + 1}{B_p - A_p * \xi_p - C_p * \xi_{p+1}^-}$ 
		\end{enumerate}
	\end{enumerate}

\end{document}