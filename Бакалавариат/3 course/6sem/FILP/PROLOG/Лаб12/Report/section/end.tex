\documentclass[../main.tex]{subfile}
\begin{document}
	Программа на Prolog состоит из разделов. Каждый раздел начинается со своего заголовка.
	Структура программы: 
	\begin{enumerate}
		\item Директивы компилятора - зарезервированные символьные константы
		\item CONSTANTS — раздел описания констант
		\item DOMAINS — раздел описания доменов
		\item DATABASE — раздел описания предикатов внутренней базы данных
		\item PREDICATES — раздел описания предикатов
		\item CLAUSES — раздел описания предложений базы знаний
		\item GOAL — раздел описания внутренней цели (вопроса).
	\end{enumerate}
	В программе не обязательно должны быть все разделы.
	
	Основными объектами программы на prolog являются: Предикаты, правила, цели.
\end{document}